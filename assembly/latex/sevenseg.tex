
\documentclass[a4paper,12pt]{article}
\usepackage{amsmath,amssymb}
\usepackage{geometry}
\usepackage{enumitem}
\usepackage{fancyhdr}
\usepackage{xcolor}
\usepackage{sectsty}
\usepackage{graphicx}
\usepackage{array}
\usepackage{float}

\geometry{top=1in,bottom=1.5cm,left=1.5cm,right=1.5cm}
\pagestyle{empty}
\sectionfont{\color{blue}}

\begin{document}

\thispagestyle{fancy}
\fancyhf{}
\fancyhead[L]{\includegraphics[width=0.3\textwidth]{image/iiitb_logo.png}}
\fancyhead[R]{
Name: Pirjade Sabahat Rehan \\
Batch: COMETFWC056 \\
Date: 22 February 2026
}
\renewcommand{\headrulewidth}{0pt}
\fancyfoot[C]{\thepage}

\vspace*{4cm}
\begin{center}
{\Huge \textbf{\textcolor{blue}{GATE IN 2009 -- Question 8 Complete Analysis}}}
\end{center}

\section*{Question 8}

\textbf{Q.8} The minimal sum-of-products expression for the logic function $f$ represented by the given Karnaugh map is:

\vspace{0.5cm}

\begin{center}
\renewcommand{\arraystretch}{1.6}
\setlength{\tabcolsep}{14pt}
\begin{tabular}{c|cccc}
 & \multicolumn{4}{c}{\textbf{PQ}} \\
\textbf{RS} & 00 & 01 & 11 & 10 \\
\hline
00 & 0 & 1 & 0 & 0 \\
01 & 0 & 1 & 1 & 1 \\
11 & 1 & 1 & 1 & 0 \\
10 & 0 & 0 & 1 & 0 \\
\end{tabular}
\end{center}

\vspace{0.8cm}

\begin{minipage}{0.45\textwidth}
(A) $QS + PR'S + PQR + P'RS + P'QR'$ \\[0.3cm]
(B) $Q'S + P'RS + P'Q'R + PR'S + PQ'R$
\end{minipage}
\hfill
\begin{minipage}{0.45\textwidth}
(C) $P'RS' + P'Q'R + PR'S + PQ'R$ \\[0.3cm]
(D) $PR'S + PQR + P'RS + P'QR'$
\end{minipage}

\vspace{0.7cm}

\textbf{Answer: (A) $QS + PR'S + PQR + P'RS + P'QR'$}

\begin{flushright}
(GATE IN 2009)
\end{flushright}

\section*{Question Analysis}

\begin{itemize}

\item The given problem involves a four-variable logic function $f(P,Q,R,S)$ represented using a Karnaugh map.

\item The variables $P$ and $Q$ form the column indices, while $R$ and $S$ form the row indices, arranged in Gray code order $00,01,11,10$.

\item Each cell containing a value $1$ represents a minterm where the function output is equal to $1$.

\item The objective is to obtain the minimal Sum-of-Products (SOP) expression by grouping adjacent 1’s.

\item Valid groupings must be formed in powers of two such as 1, 2, or 4 cells.

\item While forming groups, variables that change within the group are eliminated, and only constant variables remain in the product term.

\item After systematic grouping of all possible 1’s, the minimized expression becomes:

$f = QS + PR'S + PQR + P'RS + P'QR'$.

\item This expression represents the minimal realization of the given logic function.

\end{itemize}

\section*{Required Components and Connections}

\begin{minipage}{0.45\textwidth}
\begin{center}
\begin{tabular}{|c|l|}
\hline
S.No & Component \\ \hline
1 & Arduino Uno Board \\
2 & Breadboard \\
3 & Seven Segment Display \\
4 & 220$\Omega$ Resistor \\
5 & Jumper Wires \\
6 & USB Cable \\
\hline
\end{tabular}
\end{center}
\end{minipage}
\hspace{0.05\textwidth}
\begin{minipage}{0.45\textwidth}
\begin{center}
\begin{tabular}{|c|c|}
\hline
Segment & Arduino Pin \\ \hline
a & 2 \\
b & 3 \\
c & 4 \\
d & 5 \\
e & 6 \\
f & 7 \\
g & 8 \\
VCC & 5V \\
GND & GND \\
\hline
\end{tabular}
\end{center}
\end{minipage}

\section*{Logic Description}

\begin{itemize}

\item The logic function is implemented using AND-OR realization based on the simplified SOP expression.

\item Each product term such as $QS$, $PR'S$, and $PQR$ is implemented using AND operations.

\item Complemented variables $P'$ and $R'$ are generated using NOT operations.

\item All product terms are finally combined using OR operations to produce the output $f$.

\item The final Boolean expression ensures minimal hardware complexity.

\end{itemize}


\section*{Code Uploading Steps}
\begin{enumerate}
        \item Create a new folder for BLINK
        \item Write The code in main.asm
        \item Run the main.asm with command "avra main.asm. It will compile the code and creates .hex file
        \item Copy the .hex file to ArduinoDriod folder
        \item connect the Arduino UNO to mobile with OTG cable
        \item Upload the hex file using "upload precomplied" option
        \item Observe the ouput and verify the expression
\end{enumerate}

\section*{Experimental Truth Table}

\begin{center}
\renewcommand{\arraystretch}{1.5}
\setlength{\tabcolsep}{12pt}
\begin{tabular}{|c|c|c|c|c|}
\hline
P & Q & R & S & f \\
\hline
0 & 0 & 0 & 0 & 0 \\
0 & 0 & 0 & 1 & 0 \\
0 & 0 & 1 & 0 & 0 \\
0 & 0 & 1 & 1 & 1 \\
0 & 1 & 0 & 0 & 1 \\
0 & 1 & 0 & 1 & 1 \\
0 & 1 & 1 & 0 & 1 \\
0 & 1 & 1 & 1 & 1 \\
1 & 0 & 0 & 0 & 0 \\
1 & 0 & 0 & 1 & 0 \\
1 & 0 & 1 & 0 & 0 \\
1 & 0 & 1 & 1 & 1 \\
1 & 1 & 0 & 0 & 0 \\
1 & 1 & 0 & 1 & 1 \\
1 & 1 & 1 & 0 & 0 \\
1 & 1 & 1 & 1 & 1 \\
\hline
\end{tabular}
\end{center}
\begin{center}
\includegraphics[width=0.4\textwidth]{image/connection.jpg}
\end{center}
\section*{Conclusion}

\begin{itemize}

\item The Karnaugh map was analyzed systematically to derive the minimal SOP expression.

\item Proper grouping of adjacent 1’s eliminated redundant variables and reduced complexity.

\item The simplified expression obtained is $f = QS + PR'S + PQR + P'RS + P'QR'$.

\item Hardware implementation verified the theoretical simplification.

\item Therefore, the minimal SOP expression corresponds to option (A).

\end{itemize}

\end{document}

