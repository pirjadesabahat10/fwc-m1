\documentclass[a4paper,12pt]{article}

\usepackage{amsmath,amssymb}
\usepackage{geometry}
\usepackage{fancyhdr}
\usepackage{xcolor}
\usepackage{sectsty}
\usepackage{graphicx}
\usepackage{array}
\usepackage{float}

\geometry{top=1in,bottom=1.5cm,left=1.5cm,right=1.5cm}
\sectionfont{\color{blue}}
\pagestyle{empty}

\begin{document}

\thispagestyle{fancy}
\fancyhf{}
\fancyhead[L]{\includegraphics[width=0.2\textwidth]{images/iiitb_logo.png}}
\fancyhead[R]{
Name: Pirjade Sabahat Rehan \\
Batch: COMETFWC056 \\
Date: 22 February 2026
}
\renewcommand{\headrulewidth}{0pt}
\fancyfoot[C]{\thepage}

\vspace*{4cm}

\begin{center}
{\Huge \textbf{\textcolor{blue}{AVR-GCC 2009 EC -- Question 36 Analysis}}}
\end{center}

\section*{Question}

\textbf{Q.36)} If $X = 1$ in the logic equation  

$F = X + Z ( Y' + ( Z' + X Y' ) ) ( X' + Z' ( X + Y ) ) = 1$,  

then determine the correct condition.

\vspace{0.3cm}

\begin{minipage}{0.45\textwidth}
(A) $Y = Z$ \\[0.3cm]
(B) $Y = Z'$
\end{minipage}
\hfill
\begin{minipage}{0.45\textwidth}
(C) $Z = 1$ \\[0.3cm]
(D) $Z = 0$
\end{minipage}

\vspace{0.5cm}

\textbf{Answer: (D) $Z = 0$}

\begin{flushright}
(GATE EC 2009)
\end{flushright}

\section*{Detailed Question Analysis}

\begin{itemize}

\item The given Boolean expression is  
$F = X + Z ( Y' + ( Z' + X Y' ) ) ( X' + Z' ( X + Y ) )$.

\item The condition given is $X = 1$.

\item When $X = 1$, then $X' = 0$.

\item Substitute $X = 1$ into the expression.

\item The first part becomes $1 + Z ( Y' + ( Z' + Y' ) )$.

\item Since $1 + A = 1$, the first part reduces to $1$.

\item The second part becomes $0 + Z' ( 1 + Y )$.

\item Since $1 + Y = 1$, the second part reduces to $Z'$.

\item Therefore, the overall expression simplifies to $F = 1 \cdot Z'$.

\item Hence, $F = Z'$.

\item For $F = 1$, we must have $Z' = 1$.

\item This implies $Z = 0$.

\item Therefore, option (D) is correct.

\end{itemize}

\section*{Logic Description}

\begin{itemize}

\item The Boolean function is implemented using basic logic gates.

\item NOT gates generate the complemented variables $X'$, $Y'$, and $Z'$.

\item AND gates implement product terms such as $X Y'$ and $Z' ( X + Y )$.

\item OR gates implement summation terms such as $Z' + X Y'$ and $X + Y$.

\item The two major parts of the expression are combined using an AND gate.

\item When $X = 1$, the circuit reduces to $F = Z'$.

\item Thus, the output becomes HIGH only when $Z = 0$.

\end{itemize}

\section*{Required Components and Connections}

\begin{minipage}{0.45\textwidth}
\begin{center}
\begin{tabular}{|c|l|}
\hline
S.No & Component \\ \hline
1 & Arduino Uno Board \\
2 & Breadboard \\
3 & Seven Segment Display \\
4 & 220$\Omega$ Resistor \\
5 & Jumper Wires \\
6 & USB Cable \\
\hline
\end{tabular}
\end{center}
\end{minipage}
\hspace{0.05\textwidth}
\begin{minipage}{0.45\textwidth}
\begin{center}
\begin{tabular}{|c|c|}
\hline
Segment & Arduino Pin \\ \hline
a & Digital 2 \\
b & Digital 3 \\
c & Digital 4 \\
d & Digital 5 \\
e & Digital 6 \\
f & Digital 7 \\
g & Digital 8 \\
GND & GND \\
VCC & 5V \\
\hline
\end{tabular}
\end{center}
\end{minipage}

\section*{Code Uploading Steps}

\begin{enumerate}
\item Create project folder and write Makefile.
\item Write the program in main.c.
\item Run the command make to generate the .hex file.
\item Copy the .hex file to ArduinoDroid folder.
\item Connect Arduino UNO using OTG cable.
\item Upload the .hex file using upload precompiled option.
\item Observe output and verify logical condition.
\end{enumerate}

\section*{Truth Table for F and Z}

\begin{center}
\renewcommand{\arraystretch}{1.6}
\setlength{\tabcolsep}{18pt}
\begin{tabular}{|c|c|}
\hline
F & Z \\
\hline
1 & 0 \\
\hline
0 & 1 \\
\hline
\end{tabular}
\end{center}

\section*{Hardware Implementation}

\begin{itemize}

\item The simplified expression is $F = Z'$.

\item When $Z = 0$, the output becomes HIGH.

\item When $Z = 1$, the output becomes LOW.

\item The hardware results match the theoretical simplification.

\end{itemize}

\begin{center}
\includegraphics[width=0.6\textwidth]{images/avr-gcc.jpg}
\end{center}

\section*{Conclusion Summary}

\begin{itemize}

\item Substituting $X = 1$ simplifies the Boolean expression.

\item The expression reduces to $F = Z'$.

\item For the output to be equal to $1$, the condition $Z = 0$ must hold.

\item Both algebraic simplification and hardware verification confirm the result.

\item Therefore, option (D) is correct.

\end{itemize}

\end{document}

