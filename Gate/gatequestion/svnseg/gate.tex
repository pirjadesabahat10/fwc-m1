\documentclass[a4paper,12pt]{article}
\usepackage{amsmath, amssymb}
\usepackage{geometry}
\usepackage{enumitem}
\usepackage{fancyhdr}
\usepackage{xcolor}
\usepackage{sectsty}
\usepackage{multicol}
\usepackage{graphicx}
\usepackage{float}
\def\inputGnumericTable{}
\usepackage[latin1]{inputenc}
\usepackage{fullpage}
\usepackage{color}
\usepackage{array}
\usepackage{longtable}
\usepackage{calc}
\usepackage{multirow}
\usepackage{hhline}
\usepackage{ifthen}
\geometry{top=1in, bottom=1.5cm, left=1.5cm, right=1.5cm}
\pagestyle{empty}
\sectionfont{\color{blue}}

\begin{document}

\thispagestyle{fancy}
\fancyhf{}
\fancyhead[L]{
\includegraphics[width=0.2\textwidth]{images/iiitb_logo.png}

}
\fancyhead[R]{
    Name: Pirjade Sabahat Rehan \\
    Batch: COMETFWC056 \\
    Date: 16 February 2026
}
\renewcommand{\headrulewidth}{0pt}
\fancyfoot[C]{\thepage}
\vspace*{1cm}

\begin{center}
	{\Huge \textbf{\textcolor{blue}{GATE 2009, EC, $60^{th}$ Question Analysis}}}
\end{center}

\section*{\textbf{Question 60}}

\begin{flushleft}
60) What are the minimum numbers of NOT gates and 2-input OR gates required to design the logic of the driver for this 7-segment display?
\end{flushleft}

\vspace{0.5cm}

\noindent
\begin{minipage}{0.45\textwidth}
(a) 3 NOT and 4 OR \\[0.3cm]
(b) 2 NOT and 4 OR
\end{minipage}
\hfill
\begin{minipage}{0.45\textwidth}
(c) 1 NOT and 3 OR \\[0.3cm]
(d) 2 NOT and 3 OR
\end{minipage}

\vspace{0.5cm}

\begin{flushright}
(GATE EC 2009)
\end{flushright}

\section*{\textbf{Question Analysis:}}

\begin{itemize}
\item Let the 4-bit BCD input be $D, C, B, A$ where $A$ is the LSB.
\item Let the output of the required 7-segment driver be $F$.
\item The segment should glow for decimal inputs $0,2,3,5,6,7,8,9$ and remain OFF for $1,4$.
\item The canonical SOP expression is $F = \sum m(0,2,3,5,6,7,8,9)$.
\item The don't care conditions are $d = \sum m(10,11,12,13,14,15)$.
\item After Karnaugh map simplification, the minimized expression is $F = A + B + \overline{C} + \overline{D}$.
\item For NOT gates, $\overline{C}$ and $\overline{D}$ are required.
\item For OR gates, the implementation can be written as $F = (A + B) + (\overline{C} + \overline{D})$.
\item Hence, the minimum requirement is $2$ NOT gates and $3$ two-input OR gates.
\end{itemize}

\newpage

\section*{\textbf{The Truth Table}}

\section*{\textbf{Hardware Implementation}}

The above problem is implemented and tested in hardware using Arduino UNO board. Here we used a seven segment display, and inputs A B C D to display output F is 1 or 0 as per truth table and verified the expression.

\section*{Required Components \& Pin Connections}

\begin{center}
\begin{minipage}{0.45\textwidth}
\begin{table}[H]
\centering
\begin{tabular}{|c|l|}
\hline
\textbf{S.No} & \textbf{Component} \\ \hline
1 & Arduino Uno Board \\
2 & Breadboard \\
3 & Seven segment (1) \\
4 & Resistors: 220$\Omega$ (1) \\
5 & Jumper Wires \\
6 & USB Cable \\
\hline
\end{tabular}
\end{table}
\end{minipage}
\hspace{0.05\textwidth}
\begin{minipage}{0.45\textwidth}
\begin{table}[H]
\centering
\begin{tabular}{|c|c|}
\hline
\textbf{Component} & \textbf{Arduino Pin} \\ \hline
Output a (seg a) & Digital 2 \\
Output b (seg b) & Digital 3 \\
Output c (seg c) & Digital 4 \\
Output d (seg d) & Digital 5 \\
Output e (seg e) & Digital 6 \\
Output f (seg f) & Digital 7 \\
Output g (seg g) & Digital 8 \\
GND & GND \\
VCC & 5V \\
\hline
\end{tabular}
\end{table}
\end{minipage}
\end{center}

\section*{Logic Description}

\begin{itemize}
\item Let the 4-bit BCD inputs be $D, C, B, A$.
\item Initialize the inputs as $D=0, C=0, B=0, A=0$.
\item Let the output of the required 7-segment driver be $F$.
\item From the experimental truth table, the segment should glow for $0,2,3,5,6,7,8,9$ and remain OFF for $1,4$.
\item The canonical expression obtained from the truth table is $F = \sum m(0,2,3,5,6,7,8,9)$.
\item The don't care conditions are $d = \sum m(10,11,12,13,14,15)$.
\item After Karnaugh map simplification, the minimized output expression is $F = A + B + C' + D'$.
\item Implement the logic using two NOT operations to generate $C'$ and $D'$.
\item Combine the terms using three two-input OR gates as $F = (A + B) + (C' + D')$.
\item Change the inputs as per the truth table and observe the segment display.
\item The segment glows when $F = 1$ and remains OFF when $F = 0$.
\end{itemize}

\section*{Code Uploading Steps}

\begin{enumerate}
\item Create a Platform IO project
\item Write the code in main.cpp in src
\item Run the PIO project with command "pio run". It will compile the code and create .hex file
\item Copy the .hex file to ArduinoDroid folder
\item Connect the Arduino UNO to mobile with OTG cable
\item Upload the hex file using "upload precompiled" option
\item Observe the output and verify the expression
\end{enumerate}

\section*{Experimental Truth Table}

\begin{center}
\renewcommand{\arraystretch}{1.8}
\setlength{\tabcolsep}{12pt}
\Large
\begin{tabular}{|c|c|c|c|c|c|}
\hline
\textbf{Decimal} & \textbf{D} & \textbf{C} & \textbf{B} & \textbf{A} & \textbf{F (Output)} \\ 
\hline
0  & 0 & 0 & 0 & 0 & 1 \\ \hline
1  & 0 & 0 & 0 & 1 & 0 \\ \hline
2  & 0 & 0 & 1 & 0 & 1 \\ \hline
3  & 0 & 0 & 1 & 1 & 1 \\ \hline
4  & 0 & 1 & 0 & 0 & 0 \\ \hline
5  & 0 & 1 & 0 & 1 & 1 \\ \hline
6  & 0 & 1 & 1 & 0 & 1 \\ \hline
7  & 0 & 1 & 1 & 1 & 1 \\ \hline
8  & 1 & 0 & 0 & 0 & 1 \\ \hline
9  & 1 & 0 & 0 & 1 & 1 \\ \hline
10 & 1 & 0 & 1 & 0 & X \\ \hline
11 & 1 & 0 & 1 & 1 & X \\ \hline
12 & 1 & 1 & 0 & 0 & X \\ \hline
13 & 1 & 1 & 0 & 1 & X \\ \hline
14 & 1 & 1 & 1 & 0 & X \\ \hline
15 & 1 & 1 & 1 & 1 & X \\ \hline
\end{tabular}
\end{center}

\vspace{0.8cm}

\begin{center}
\includegraphics[width=0.6\textwidth]{images/connection.jpg}

\end{center}

\section*{Conclusion}

\begin{itemize}
\item From the experimental truth table, $F = 1$ for decimal inputs $0,2,3,5,6,7,8,9$ and $F = 0$ for $1,4$.
\item The minimized Boolean expression obtained is $F = A + B + C' + D'$.
\item The logic implementation requires $2$ NOT gates and $3$ two-input OR gates.
\item This matches option (D) from the original GATE question.
\item The hardware experiment confirms the theoretical logic of the 7-segment driver.
\end{itemize}

\end{document}
